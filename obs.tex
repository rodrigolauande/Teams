- criando o projeto --> npx create-expo-app --template

- iniciando o projeto --> npx expo start

- mapeamento de importações, --> npm install --save-dev babel-plugin-module-resolver
- configuração "babel.config.js e tsconfig.json"

- instalação 'styled components e suas typagems"
- npm i styled-components
- npm i --save-dev @types/styled-components
- npm i --save-dev @types/styled-components-react-native

- definição do thema da aplicação "theme/index.ts
- configuração do thema no app da aplicação usando o 'themeProvider do styled components'
- definição das typagems para usar o thema na aplicação '@types/styled.d.ts'

- instalação das fontes"
- npm i expo-font
- npm i @expo-google-fonts/roboto 
- importa no app -> import {useFonts, Roboto_400Regular, Roboto_700Bold } from '@expo-google-fonts/roboto'
- configuraçao do estado que carrega as fontes configuradas const [fontsLoaded] = useFonts([ Roboto_400Regular, Roboto_700Bold]);

- Header
- definição das typagems para usar imagens na aplicação '@types/png.d.ts'.

- usando icones na aplicação "phosphor-react-native"
- npm i --save phosphor-react-native
- npm i react-native-svg


- resolvedo conflito de typescript
- define a configuraçao no package.json
- "overrides": {
    "@types/react": "~18.2.14",
    },
    - salva, exclua a pasta node modules, e instala o projeto de novo "npm install"
    
- Navegação
- Stack Navigator  "Cora da aplicação" -> npm install @react-navigation/native
- dependencias que precisa lidar com a parte da navegação e suguras da aplicação -> npx expo install react-native-screens react-native-safe-area-context
- extrategia de naveção que será utilizada -> npm install @react-navigation/native-stack

- Tipagem para navegação
- @types/navigation.ds

- SafeAreaView
- cuida do container da View na aplicação
- export const Container = styled(SafeAreaView)``;

-AsyncStorange - Local Storange "Armazenamento no dispositivo"
- npm i @react-native-async-storage/async-storage

-UseFocusEffect - redenrizar a chamada API, para funcionar necessita usar o UseCallback
- useFocusEffect(useCallback(() => {
    console.log('useFocusEffect Execultou')
    fetchGroups();
  },[]));


